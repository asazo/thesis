%%%%%%%%%%%%%%%%%%%%%%%%%%%%%%%%%%%%%%%%%%%%%%%%%%%%%%%%%%%%%%%%%%%%%%%%%%%%%
%
% Macros for writing recurrent and complex math expressions,
% mostly related to grain structure concepts such as the grain boundaries \xi
% and its spatial and temporal derivatives.
% One of the mathematical models also requires to write over and over interpolators.
% this is handled too.
%

% Write complex commands, better than \newcommand
\usepackage{xparse}
% Vertical scaling
\usepackage{scalerel}
% Bold typesetting, in general, but useful for make greek letters bold
\usepackage{bm}

%%% MATH
% Real set
\newcommand{\Reals}{\mathbb{R}}
% Rectangular domain over R^2
\newcommand{\dom}{[0,1]^2 \subset \mathbb{R}^2}
% Complex set
\newcommand{\Complex}{\mathbb{C}}
% sinc
\DeclareMathOperator\sinc{sinc}
% number of sides
\DeclareMathOperator\ns{ns}
% Norm
\newcommand{\norm}[1]{\left\lVert#1\right\rVert}
% Vectores en negrita
\renewcommand{\vec}[1]{\mathbf{#1}}

%%% GRAINS COMMANDS

% Grains set
\newcommand{\grn}{\mathcal{G}}
\newcommand{\grains}{\bm{\mathcal{G}}}
% Boundaries set
\newcommand{\bnd}{\Gamma}
\newcommand{\boundaries}{\bm{\bnd}}
% Vertices set
\newcommand{\vertices}{\bm{\mathcal{X}}}

% Vectorial notation
% #1 objective vector
% #2 superscript with parenthesis
% #3 underscript without parenthesis
\DeclareDocumentCommand \vectorial { m o o}{
    \IfNoValueTF{#3}{
        \IfNoValueTF{#2}{
            \vec{#1}
        }{
          \vec{#1}_{#2}^{\phantom{()}}\!\!
        }
    }{
	  \vec{#1}_{#2}^{(#3)}
    }
}

% discrete data
\DeclareDocumentCommand \x { o o }{ \vectorial{x}[#1][#2] }
% discrete data
\DeclareDocumentCommand \y { o o }{ \vectorial{y}[#1][#2] }
% Capital bold X
\newcommand{\X}{\vectorial{X}}
% dot x
\DeclareDocumentCommand \dotx { o o }{ \vectorial{\dot{x}}[#1][#2] }
% dot y
\DeclareDocumentCommand \doty { o o }{ \vectorial{\dot{y}}[#1][#2] }
% Canonical vector
\newcommand{\ei}[1]{\mathbf{e}_{#1}}
% xi boundary
\newcommand{\vxi}{\bm{\xi}}
% l(s,t)
\newcommand{\mylvec}{\vec{l}}
% derivative of xi with respect to s
\newcommand{\dxids}{\dfrac{\partial \vxi}{\partial s}}
% derivative of xi with respect to t
\newcommand{\dxidt}{\dfrac{\partial \vxi}{\partial t}}
% Tangent vector
\newcommand{\T}{\vec{T}}
% Hat tangent vector
\newcommand{\hatT}{\widehat{\vec{T}}}
% Normal vector
\newcommand{\N}{\vec{N}}
% Hat normal vector
\newcommand{\hatN}{\widehat{\vec{N}}}
% Rate of change area
\newcommand{\dAdt}{\dfrac{dA}{dt}}
% Velocity vector
\newcommand{\vel}{\vec{v}}
% Explicit tangent definition as unit vector
\newcommand{\unitl}{\dfrac{\mylvec(s,t)}{\norm{\mylvec(s,t)}}}
\newcommand{\unitlk}{\dfrac{\mylvec^{(k)}(s,t)}{\norm{\mylvec^{(k)}(s,t)}}}
% Derivative of tangent with respect to s
\newcommand{\dTds}{\dfrac{\partial \T}{\partial s}}
% Derivative of l(s,t) with respect to t
\newcommand{\dlvecdt}{\dfrac{\partial \mylvec}{\partial t}}
\newcommand{\dlkvecdt}{\dfrac{\partial \mylvec^{(k)}}{\partial t}}
% Derivative of velocity with respect to space
\newcommand{\dvds}{\dfrac{\partial \vel}{\partial s}}
% Standalone d/dt
\newcommand{\partddt}{\dfrac{\partial}{\partial t}}
% Standalone d/ds
\newcommand{\partdds}{\dfrac{\partial}{\partial s}}
% Evaluate an expression between some interval
\newcommand{\eval}[2]{\bigg\rvert_{#1}^{#2}}
% Abreviature, useful to declare AL(s)
\newcommand{\AL}{\mathcal{L}}
% Derivative of T with respect to arclength
\newcommand{\dTdAL}{\dfrac{\partial \T}{\partial \AL}}
% Derivative of s with respect to arclength
\newcommand{\dsdAL}{\dfrac{ds}{d\AL}}
% Lagrange phi function, with space to fig with x_a^b
% by adding a phantom exponent
\newcommand{\phii}[2]{\phi_{#1}^{\phantom{()}}\!\!\!\left(#2\right)}
% Boundary definition
\newcommand{\boundary}{ \sum_{i=1}^{n} \x[i][k](t)\,\phii{i}{s}}
% Boundary definition 2
\newcommand{\boundarytwo}{ \sum_{i=1}^{n} \x[i](t)\,\phii{i}{s}}
% Velocity boundary
\newcommand{\velboundary}{ \sum_{i=1}^{n} \dotx[i](t)\,\phii{i}{s}}
% Stored energy
\newcommand{\SE}{\mathcal{E}}

%% LATIN LOCUTION
\newcommand{\ie}{i.e.,\;}