\chapter{Abstract}

The microstructure of polycrystalline materials are composed by grains separated by their interfaces called grain boundaries, meeting at triple junctions. The orientation and shape of these grains define material properties such as resistance, electric conductivity, among others. The improvement of such properties is achieved by modifying the underlying structure via primary recrystallization and grain growth. In this work we analyze in depth and implement two-dimensional and three-dimensional grain growth and nucleation models and obtain relevant statistics. We deal with computational challenges related to algorithms scalability and parallel programming in GPU with the objective to be able to simulate hundreds of thousands of grains, for example the improvement of the extinction time estimation and flipping detection and the parallel management of topological transitions in two-dimensional models. We finally extract statistics from another three-dimensional model with image analysis techniques.

\chapter{Resumen}

La estructura interna de los materiales policristalinos está compuesta por granos, separados por sus interfaces llamadas fronteras, las cuales inciden en uniones triples. La orientación y la forma de estos granos definen propiedades de los materiales tales como resistencia, conductividad eléctrica entre otras. Para mejorar estas propiedades se debe modificar la estructura subyacente de granos vía recristalizacion y crecimiento de granos. En este trabajo se analizan en profundidad e implementan modelos de nucleacion y de crecimiento de granos en dos y tres dimensiones y se obtienen estadísticas relevantes. Además, se abordan retos computacionales relacionados a la escalabilidad de los algoritmos y programación paralela en GPU con el objetivo de simular cientos de miles de granos, por ejemplo la mejora de la estimación del tiempo de extinción y de detección de flippings, y el manejo en paralelo de transiciones topológicas en modelos de dos dimensiones. Finalmente se extraen estadisticas de otro modelo tridimensional con técnicas de análisis de imágenes.
